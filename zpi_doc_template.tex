\documentclass[a4paper]{article}
\usepackage{polski}
\usepackage[cp1250]{inputenc}
\usepackage{url}


\newcommand{\norm}[1]{\left\Vert#1\right\Vert}

\title{\bf{<<Tytuł projektu>>}}
\author{{\em <<Autorzy>>}}
\date{}

\begin{document}

\begin{titlepage}
\maketitle
\thispagestyle{empty}
\bigskip
\begin{center}
Zespołowe przedsięwzięcie projektowe\\[2mm]

Informatyka\\[2mm]

Rok. akad. 2017/2018, sem. I\\[2mm]

Prowadzący: dr hab. Marcin Mazur
\end{center}
\end{titlepage}

\tableofcontents
\thispagestyle{empty}

\newpage

<<Uwaga: przed rozpoczęciem realizacji projektu warto poczytać {\em Scrum Guide} (zob. \cite{SchSut}).>>

\section{Opis projektu}

\subsection{Członkowie zespołu}

\begin{enumerate}
\item <<Imię i nazwisko>> (kierownik projektu).
\item <<Imię i nazwisko>>.
\item <<Imię i nazwisko>>.
\item <<Imię i nazwisko>>.
\end{enumerate}

\subsection{Cel projektu (produkt)}

<<Krótko opisać, jaki jest cel realizowanego projektu (określić uzyskiwany produkt)>>.

\subsection{Potencjalny odbiorca produktu (klient)}

<<Określić potencjalnego klienta (wraz z uzasadnieniem)>>.

\subsection{Metodyka}

Projekt będzie realizowany przy użyciu (zaadaptowanej do istniejących warunków) metodyki {\em Scrum}. 

\section{Wymagania użytkownika}
<<Przedstawić listę wymagań użytkownika w postaci ,,historyjek'' (User stories). Każda historyjka powinna opisywać jedną cechę systemu.>>

\subsection{User story 1}
<<Historyjka 1>>.

\subsection{User story 2}
<<Historyjka 2>>.

\subsection*{<<Tutaj dodawać kolejne historyjki>>}

\section{Harmonogram}

\subsection{Rejestr zadań (Product Backlog)}

\begin{itemize}
\item Data rozpoczęcia: <<data>>.
\item  Data zakończenia: <<data>>.
\end{itemize}

\subsection{Sprint 1}

\begin{itemize}
\item Data rozpoczęcia: <<data>>.
\item Data zakończenia: <<data>>.
\item Scrum Master: <<imię i nazwisko>>.
\item Product Owner: <<imię i nazwisko>>.
\item Developerzy: <<lista developerów>>.
\end{itemize}

\subsection{Sprint 2}

\begin{itemize}
\item Data rozpoczęcia: <<data>>.
\item  Data zakończenia: <<data>>.
\item Scrum Master: <<imię i nazwisko>>.
\item Product Owner: <<imię i nazwisko>>.
\item Developerzy: <<lista developerów>>.
\end{itemize}

\subsection*{<<Tutaj dodawać kolejne sprinty>>}

\section{Product Backlog}

\subsection{Product Item 1}
\paragraph{Etykieta zadania.} <<Etykieta>>.
\paragraph{Opis zadania.} <<Opis>>.
\paragraph{Priorytet.} <<Priorytet>>.
\paragraph{Definition of Done.} <<Określić (w języku zrozumiałym dla wszystkich członków zespołu), co oznacza ukończenie danego zadania>>.

\subsection{Product Item 2}
\paragraph{Etykieta zadania.} <<Etykieta>>.
\paragraph{Opis zadania.} <<Opis>>.
\paragraph{Priorytet.} <<Priorytet>>.
\paragraph{Definition of Done.} <<Określić (w języku zrozumiałym dla wszystkich członków zespołu), co oznacza ukończenie danego zadania>>.

\subsection*{<<Tutaj dodawać kolejne zadania>>}

\section{Sprint 1}
\subsection{Cel} <<Określić, w jakim celu tworzony jest przyrost produktu>>.
\subsection{Sprint Planning/Backlog}

\paragraph{Etykieta zadania.} <<Etykieta>>.
\begin{itemize}
\item Czasochłonność: <<szacowana czasochłonność>>.
\item{Ryzyko:} <<określić ryzyko zadania>>.
\end{itemize}

\paragraph{Etykieta zadania.} <<Etykieta>>.
\begin{itemize}
\item Czasochłonność: <<szacowana czasochłonność>>.
\item{Ryzyko:} <<określić ryzyko zadania>>.
\end{itemize}

\paragraph{<<Tutaj dodawać kolejne zadania>>}

\subsection{Realizacja}

\paragraph{Etykieta zadania.} <<Etykieta>>.
\subparagraph{Wykonawca.} <<Wykonawca>>.
\subparagraph{Realizacja.} <<Sprawozdanie z realizacji zadania>>.

\paragraph{Etykieta zadania.} <<Etykieta>>.
\subparagraph{Wykonawca.} <<Wykonawca>>.
\subparagraph{Realizacja.} <<Sprawozdanie z realizacji zadania>>.

\paragraph{<<Tutaj dodawać kolejne zadania>>}


\subsection{Sprint Review/Demo}
<<Sprawozdanie z przeglądu Sprint'u -- czy założony cel (przyrost) został osiągnięty?>>.

\section{Sprint 2}

\subsection{Cel} <<Określić, w jakim celu tworzony jest przyrost produktu>>.

\subsection{Sprint Planning/Backlog}

\paragraph{Etykieta zadania.} <<Etykieta>>.
\begin{itemize}
\item Czasochłonność: <<szacowana czasochłonność>>.
\item{Ryzyko:} <<określić ryzyko zadania>>.
\end{itemize}

\paragraph{Etykieta zadania.} <<Etykieta>>.
\begin{itemize}
\item Czasochłonność: <<szacowana czasochłonność>>.
\item{Ryzyko:} <<określić ryzyko zadania>>.
\end{itemize}

\paragraph{<<Tutaj dodawać kolejne zadania>>}

\subsection{Realizacja}

\paragraph{Etykieta zadania.} <<Etykieta>>.
\subparagraph{Wykonawca.} <<Wykonawca>>.
\subparagraph{Realizacja.} <<Sprawozdanie z realizacji zadania>>.

\paragraph{Etykieta zadania.} <<Etykieta>>.
\subparagraph{Wykonawca.} <<Wykonawca>>.
\subparagraph{Realizacja.} <<Sprawozdanie z realizacji zadania>>.

\paragraph{<<Tutaj dodawać kolejne zadania>>}


\subsection{Sprint Review/Demo}
<<Sprawozdanie z przeglądu Sprint'u -- czy założony cel (przyrost) został osiągnięty?>>.

\section*{<<Tutaj dodawać kolejne zadania>>}

\begin{thebibliography}{9}

\bibitem{Cov} S. R. Covey, {\em 7 nawyków skutecznego działania}, Rebis, Poznań, 2007.

\bibitem{SchSut} K. Schwaber, J. Sutherland, {\em Scrum Guide}, \url{http://www.scrumguides.org/}, 2016.

\end{thebibliography}
\end{document}

% ----------------------------------------------------------------
